\documentclass[a4paper]{article}

\usepackage[margin=2cm]{geometry}
\usepackage{fontspec}
\usepackage[catalan]{babel}

\usepackage{amsfonts, amsmath}
\usepackage{cancel}
\usepackage{algorithm}
\usepackage{algpseudocode}
\usepackage{algorithmicx}
\usepackage{float}

\def\und#1{\underbrace{#1}}

\begin{document}

\begin{enumerate}
\item \textbf{Calculeu:}

\begin{itemize}
	\item $3^{28} \mod 10$
	$$
	3^{28} \equiv_{10} (3^{4})^7 \equiv_{10} (3 \cdot (3^{3}))^7 \equiv_{10} 
	(3 \cdot (27 \mod 10))^7 \equiv_{10} ((3 \cdot 7) \mod 10)^{7} \equiv_{10} 1^7 \equiv_{10} \boldsymbol{1}
	$$
	
	\item $3^{200} \mod 15$
	\begin{flalign*}
		3^{200} &\equiv_{15} (3^8)^{25} \equiv_{15} (3^5 \cdot 
		\und{(3^3 \mod 15)}_{12})^{50} \equiv_{15} 
		(3^2 \cdot \und{12}_{3 \cdot 4} \cdot \und{(3^3 \mod 15)}_{12})^{25} \equiv_{15} \\
		& \equiv_{15} (3 \cdot \und{3^3}_{12} \cdot (\cancelto{1}{4^2 \mod 15}))^{25} 
		\equiv_{15} 
		(3 \cdot \und{12}_{3 \cdot 4})^{25} \equiv_{15} (3^{10} \cdot 4^5)^5 \equiv_{15} 
		(3 \cdot 3^3 \cdot 3^3 \cdot 3^3 \cdot \cancelto{1}{4^2} \cdot \cancelto{1}{4^2} 
		\cdot 4)^{5} \\
		&\equiv_{15} (3 \cdot 12^3 \cdot 4)^{5} \equiv_{15} 
		(3 \cdot 3^3 \cdot \cancelto{1}{4^2} \cdot \cancelto{1}{4^2})^5 \equiv_{15}
		(3 \cdot 12)^5 \equiv_{15} \boldsymbol{6}
	\end{flalign*}
\end{itemize}
\item \textbf{2EXP modular.} Doneu un algorisme de temps polinòmic que amb entrada els enters $a$, $b$, $c$ i un nombre primer $p$ computi $a^{b^c} \mod p$.

\begin{algorithm}[H]
	\caption{Algoritme per calcular $a^{b^c} \mod p$ quan $p$ és primer}
	\begin{algorithmic}[1]		
		\Function{2EXP\_MOD}{$a, b, c, p$}
			\State $z = $ \Call{\texttt{MODEXP}}{$b, c, p - 1$}
			\State \Return \Call{\texttt{MODEXP}}{$a, z, p$}
		\EndFunction		
	\end{algorithmic}
\end{algorithm}

Això es compleix degut al \emph{Petit teorema de Fermat} el qual afirma que per qualsevol nombre $p$ primer 
$$
a^p \equiv_{p} a
$$

del teorema es dedueix que
$$
a^{p - 1} \equiv_{p} 1
$$

\item \textbf{Factorial Modular} Donats dos enters $x$ i $N$, calcula $x! \mod N$ .
\begin{enumerate}
	\item Demostreu que un enter $y$ és primer si i només si per a tot enter $x < y$ es compleix $\gcd(x!, y) = 1$.
	\item Considereu l'apartat previ per demostrar que si \textbf{Factorial Modular} fos computable en temps polinòmic, aleshores el problema de \textbf{Factoritzar} també seria computable en temps polinòmic (Recordeu \textbf{Factoritzar}: Donat un nombre enter $x$, calcula els seus factors primers.)
\end{enumerate} 

\item En un sistema criptogràfic \textbf{RSA} amb $p = 7$ i $q = 11$, troba la clau pública $(N, c)$ i la clau privada $(N, d)$ apropiades.
\begin{align*}
	p &= 7 \\
	q &= 11 \\
	N &= p \cdot q = 7 \cdot 11 = 77 \\
	\phi(N) &= (p - 1)(q - 1) = 6 \cdot 10 = 60 \\
	c &\in \mathbb{Z}_{\phi(60)} \\
	c &= 7
\end{align*}

Un cop es troba $c$ cal buscar un $d : c \cdot d \equiv_{60} 1 \implies d = 43$. Així doncs:
\begin{align*}
	P &= (60, 7) \\
	S &= (60, 43)
\end{align*}

\end{enumerate}
\end{document}