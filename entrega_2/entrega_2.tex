\documentclass[a4paper]{article}

\usepackage[margin=2cm]{geometry}
\usepackage{fontspec}
\usepackage[catalan]{babel}

\usepackage{amsfonts, amsmath}
\usepackage{cancel}
\usepackage{algorithm}
\usepackage{algpseudocode}
\usepackage{algorithmicx}
\usepackage{float}

\def\und#1{\underbrace{#1}}

\title{Problemes  d'aritmètica modular i RSA}
\author{Marc Asenjo, Arnau Canyadell i Joan Marcè}
\date{Primavera de 2017}

\begin{document}

\maketitle

\begin{enumerate}
\item \textbf{Calculeu:}

\begin{itemize}
	\item $3^{28} \mod 10$
	$$
	3^{28} = (3^{4})^7 = (81 \mod 10)^{7} = 1^7 = \boldsymbol{1} \pmod{10}
	$$
	
	\item $3^{200} \mod 15$
%	\begin{flalign*}
%		3^{200} &\equiv_{15} (3^8)^{25} \equiv_{15} (3^5 \cdot 
%		\und{(3^3 \mod 15)}_{12})^{50} \equiv_{15} 
%		(3^2 \cdot \und{12}_{3 \cdot 4} \cdot \und{(3^3 \mod 15)}_{12})^{25} \equiv_{15} \\
%		& \equiv_{15} (3 \cdot \und{3^3}_{12} \cdot (\cancelto{1}{4^2 \mod 15}))^{25} 
%		\equiv_{15} 
%		(3 \cdot \und{12}_{3 \cdot 4})^{25} \equiv_{15} (3^{10} \cdot 4^5)^5 \equiv_{15} 
%		(3 \cdot 3^3 \cdot 3^3 \cdot 3^3 \cdot \cancelto{1}{4^2} \cdot \cancelto{1}{4^2} 
%		\cdot 4)^{5} \\
%		&\equiv_{15} (3 \cdot 12^3 \cdot 4)^{5} \equiv_{15} 
%		(3 \cdot 3^3 \cdot \cancelto{1}{4^2} \cdot \cancelto{1}{4^2})^5 \equiv_{15}
%		(3 \cdot 12)^5 \equiv_{15} \boldsymbol{6}
%	\end{flalign*}
	$$3^{200} = ((3^5)^5)^8 = ((\und{243 \mod 15}_3)^5)^8 = (3^5)^8 = 3^8 = (3^4)^2 = (\und{81 \mod 15}_6)^2= 6^2 = \boldsymbol{6} \pmod{15}$$
\end{itemize}
\item \textbf{2EXP modular.} Doneu un algorisme de temps polinòmic que amb entrada els enters $a$, $b$, $c$ i un nombre primer $p$ computi $a^{b^c} \mod p$.

\begin{algorithm}[H]
	\caption{Algoritme per calcular $a^{b^c} \mod p$ quan $p$ és primer}
	\begin{algorithmic}[1]		
		\Function{2EXP\_MOD}{$a, b, c, p$}
			\State $z = $ \Call{\texttt{MODEXP}}{$b, c, p - 1$}
			\State \Return \Call{\texttt{MODEXP}}{$a, z, p$}
		\EndFunction		
	\end{algorithmic}
\end{algorithm}

Això es compleix degut al \emph{Petit teorema de Fermat} el qual afirma que per qualsevol nombre $p$ primer 
$$
a^p \equiv_{p} a
$$

del teorema es dedueix que
$$
a^{p - 1} \equiv_{p} 1
$$

\item \textbf{Factorial Modular} Donats dos enters $x$ i $N$, calcula $x! \mod N$ .
\begin{enumerate}
	\item Demostreu que un enter $y$ és primer si i només si per a tot enter $x < y$ es compleix $\gcd(x!, y) = 1$.
	
	\begin{align*}
		y \in \mathbb{P} &\iff \forall x | 1 < x < y : y \bmod{x} = 0 \\
		y \in \mathbb{P} &\iff \gcd(x!, y) = 1 \iff \gcd((y-1)!, y) = 1 \iff \\
		&\iff \lnot\exists i | 1 < i < \min((y-1)!, y) : y \bmod{i} = 0 \land \und{(y-1)! \bmod{i} = 0}_{\text{cert } \forall i | 1 < i < y} \iff \\
		& \iff \forall i | 1 < i < y : y \bmod{i} = 0 \iff y \in \mathbb{P}
	\end{align*}
	
	\item Considereu l'apartat previ per demostrar que si \textbf{Factorial Modular} fos computable en temps polinòmic, aleshores el problema de \textbf{Factoritzar} també seria computable en temps polinòmic (Recordeu \textbf{Factoritzar}: Donat un nombre enter $x$, calcula els seus factors primers.)
	
	\begin{algorithm}[H]
		\caption{Algoritme per factoritzar a partir de Factorial Modular}
		\begin{algorithmic}[1]	
			\State $factors := \{\}$	
			\Function{Factoritzar}{$x$}
			\State $x\_fact\_mod = $ \Call{\texttt{Factorial Modular}}{$x-1, x$}
			\State $gcd = $ \Call{\texttt{Euclid}}{$x\_fact\_mod, x$}
			\If {$gcd \neq 1$}
			\State $factors.push (x / gcd)$
			\State \Call{Factoritzar}{gcd}
			\EndIf
			\State \Return
			\EndFunction		
		\end{algorithmic}
	\end{algorithm}
	
	Si \textbf{Factorial Modular} fos computable en temps polinòmic, i sabent que l'algoritme d'Euclid per trobar el GCD també ho és, només falta demostrar que la quantitat de repeticions de la recursivitat que s'executaran és també polinòmica. A cada iteració s'extreu un factor de la factorització de x, així que com a molt, si tots els factors són 2, hi haurà $log_2(x) = |x|$ repeticions, que resulten en un cost polinòmic.
	
\end{enumerate} 

\item En un sistema criptogràfic \textbf{RSA} amb $p = 7$ i $q = 11$, troba la clau pública $(N, c)$ i la clau privada $(N, d)$ apropiades.
\begin{align*}
	p &= 7 \\
	q &= 11 \\
	N &= p \cdot q = 7 \cdot 11 = 77 \\
	\phi(N) &= (p - 1)(q - 1) = 6 \cdot 10 = 60 \\
	c &\in \mathbb{Z}_{\phi(60)} \\
	c &= 7
\end{align*}

Un cop es troba $c$ cal buscar un $d : c \cdot d \equiv_{60} 1 \implies d = 43$. Així doncs:
\begin{align*}
	P &= (60, 7) \\
	S &= (60, 43)
\end{align*}

\item \textbf{Sistema criptogràfic segur?} Suposem que en lloc d'utilitzar un nombre compost $N = pq$ com es fa en el sistema \texttt{RSA}, utilitzem un nombre primer $p$. Per encriptar un missatge $m \mod p$ farem servir un exponent $e$, de la mateixa manera que es fa en el sistema \texttt{RSA}. L'encriptació del missatge $m \mod p$ seria $m^e \mod p$.

Demostreu que aquest nou sistema no és segur donant un algorisme eficient per
desencriptar. És a dir, doneu un algorisme que, amb entrada $p$, $e$, $m^e mod p$, computi $m mod p$ eficientment. Justifica la correctesa de l'algorisme i analitza el seu temps de computació.

Utilitzant l'algoritme euclidià estès (extended Euclidian algorithm) podem trobar l'invers d'un nombre en aritmètica modular en temps $O(n^3)$. Suposant que aquesta funció es diu \texttt{invers(nombre, mòdul)}, llavors un algoritme que desencriptaria el missatge $m^e$ seria el següent:

\begin{algorithm}[H]
	\caption{Algorisme desencriptador}
	\begin{algorithmic}
		\Function{desencriptar}{$p$, $e$, $m^e \mod p$}
			\State $d :=$ \Call{\textsc{invers}}{$e$, $p-1$}
			\State \Return $(m^e \mod p)^d \mod p$
		\EndFunction
	\end{algorithmic}
\end{algorithm}

\textbf{Prova de la correctesa de l'algorisme.}

\begin{align*}
	d := invers(e, p-1) &\pmod{p-1} \\
	e d = p = 1 &\pmod{p-1} \\
	(m^e)^d = m^{e d} = m^{k(p-1)+1} = m^{k(p-1)} m = m &\pmod{p}
\end{align*}

On $k \in \mathbb{Z}_+$

\end{enumerate}
\end{document}