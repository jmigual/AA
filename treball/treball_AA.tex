\documentclass[a4paper]{article}

\usepackage[margin=2cm]{geometry}
\usepackage{fontspec}
\usepackage[catalan]{babel}
\usepackage[
backend=biber,
citestyle=numeric  
]{biblatex}
\usepackage{url}

\usepackage{amsmath, amsfonts}

\usepackage{tikz}
\usetikzlibrary{positioning}
\tikzset{main node/.style={circle,fill=blue!20,draw,minimum size=1cm,inner sep=0pt},
}

\setlength{\parindent}{0pt}

\nocite{*}
\bibliography{treball_AA.bib}

\title{Entrega puntuable d'Ampliació d'Algorísmia \\ \texttt{Maximum planar subgraph}}
\author{Marc Asenjo i Ponce de León \and Arnau Canyadell i Miquel \and Joan Marcè i Igual}
\date{}

\begin{document}

\maketitle

\section{Descripció del problema}
El problema del \texttt{Maximum planar subgraph} es defineix de la manera següent:

Donat un graf $G=(V,E)$, computar un subgraf planar $G'=(V,E')$ amb el màxim nombre d'arestes.

\subsection{Exemple}
Un exemple d'instància del problema és el problema donat pel graf $G$, on $G$ és la clique de $|V|=4$. $G$ no és un subgraf planar, però es pot obtenir una solució (un subgraf planar màxim) eliminant una aresta, com es mostra en la figura següent.

\begin{figure}[!h]
	\centering
	\begin{tikzpicture}
	\node[main node] (1) {$1$};
	\node[main node] (2) [right = 2cm  of 1] {$2$};
	\node[main node] (4) [below = 2cm  of 1] {$4$};
	\node[main node] (3) [right = 2cm  of 4] {$3$};
	
	\path[draw,thick]
	(1) edge node {} (2)
	(2) edge node {} (3)
	(3) edge node {} (4)
	(4) edge node {} (1)
	(1) edge node {} (3)
	(2) edge[red] node {} (4)
	;
	\end{tikzpicture}	
\end{figure}

\section{Problemes derivats}

En aquest apartat es donen dos exemples de problemes derivats del \texttt{Maximum planar subgraph problem}. Un és un problema decisional i l'altre és un problema parametritzat.

\subsection{Problema decisional: \texttt{Is maximum planar subgraph}}
Donat un graf $G=(V,E)$ i un subgraf planar de $G$, $G'=(V,E')$, dir si $G'$ és un \texttt{Maximum planar subgraph} de $G$.

\subsection{Problema parametritzat: \texttt{K-planar subgraph}}
Donat un graf $G=(V,E)$ i un enter $k$, computar un subgraf planar $G'=(V,E')$ amb $|E'|=k$.

\section{Resultats publicats}

\subsection{Complexitat i aproximació del problema}
El problema del \texttt{Maximum planar subgraph} és un problema NP-complet. Hi ha diversos algoritmes que donen una 3-aproximació del problema\cite{betterApproximation}:
\begin{itemize}
	\item \texttt{Spanning Tree}: Trobar un spanning tree permet obtenir una 3-aproximació de \texttt{Maximum planar subgraph}. Es pot provar que el rati d'aproximació és $1/3$ perquè en un \texttt{spanning tree} amb $n$ vèrtex hi ha $n - 1$ arestes i un graf planar amb $n$ vèrtexs com a molt pot tenir $3n - 3 = 3(n - 1)$ arestes \cite{planar}.
	\item \texttt{Maximal Planar Subgraph}: que consisteix en trobar un graf planar que a l' afegir-hi una aresta deixi de ser planar.	
\end{itemize}

A part també s'han trobat algoritmes que donen un millor rati d'aproximació:
\begin{itemize}
	\item Algorisme A (versió \emph{Greedy}): té un rati de $7/18$ i està basat en produir una estructura triangular i que té cost lineal. Aquest algoritme defineix els conceptes següents:
	\begin{itemize}
		\item \emph{cactus triangular}: Graf que tots els seus cicles són triangles i totes les arestes apareixen en algun cicle.
		\item \emph{estructura triangular}: Graf que els seus cicles són triangles.
	\end{itemize}
	Així doncs l'algoritme treballa en dues etapes:
	\begin{enumerate}
		\item Produir un \emph{cactus triangular} maximal $S_1$ que sigui un subgraf de $G$.
		\item Estendre $S_1$ a una \emph{estructura triangular} $S_2$ que es trobi en $G$ afegint tantes arestes a $S_1$ com sigui possible però sense crear nous cicles.
	\end{enumerate}
	\item Algorisme B (versió millorada): té un rati de $4/9$ i té un cost de $O(m^{3/2}n\log^6 n)$. Aquest algorisme funciona segons el mateix principi que l'algorisme A però a més a més millora la primera part on es genera un \emph{cactus triangular} millorant així el resultat total de l'algoritme. 
\end{itemize}

\printbibliography

\end{document}