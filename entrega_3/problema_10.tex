\documentclass[a4paper]{article}

\usepackage[margin=2cm]{geometry}
\usepackage{fontspec}
\usepackage[english]{babel}

\usepackage{amsmath, amsfonts}

\setlength{\parindent}{0pt}

\title{Problema 10}
\author{Arnau Canyadell i Miquel \and Joan Marcè i Igual}
\date{}

\begin{document}

\maketitle

The \texttt{Set Packing Problem} is defined as follows: Given a family of sets $S_1,...,S_m \subseteq U$ such that, $\forall i \ 1 \le i \le m,\ |S_i|=3$ and has profit $c(S_i)$, find a subset of these sets that maximizes the profit, while each element is covered at most once. Consider a straightforward integer linear programming formulation of the problem:

$$
\max \sum_{i=1}^m c(S_i)x_i
$$
$$
\sum_{i|u\in S_i} x_i \le 1 \quad \forall u \in U
$$
$$
x_i \in {0, 1} \quad 1 \le i \le m
$$

Consider the following algorithm that, first computes an optimal solution $x^*$ of the LP obtained by relaxing $x_i \in [0, 1]$, and second performs the following rounding algorithm:
\begin{enumerate}
	\item Chose a set $S_i$ to be in the solution with probability $\frac{x_i^*}{6}$
	\item If an element $u\in U$ is covered by more than one set, remove all the sets in the solution that contain $u$.
\end{enumerate}

Show that the proposed algorithm is a randomized 12-approximation for \texttt{Set packing}

\textbf{Solution:}


\begin{gather*}
	x_i \in [0,1] \\
	x_i^* = P[x_{i,opt} = 1] \\
	P[x_i^1 = 1] = \frac{x_i^*}{6} = \frac{P[x_{i,opt} = 1]}{6} \\
\end{gather*}
\begin{gather*}
	P[x_i = 1] = P[x_i^1 = 1] \cdot P[x_i^2 = 1] = \frac{1}{12}
\end{gather*}

\end{document}