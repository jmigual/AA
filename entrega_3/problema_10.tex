\documentclass[a4paper]{article}

\usepackage[margin=2cm]{geometry}
\usepackage{fontspec}
\usepackage[english]{babel}

\usepackage{amsmath, amsfonts}

\setlength{\parindent}{0pt}

\title{Problema 10}
\author{Arnau Canyadell i Miquel \and Joan Marcè i Igual}
\date{}

\begin{document}

\maketitle

The \texttt{Set Packing Problem} is defined as follows: Given a family of sets $S_1,...,S_m \subseteq U$ such that, $\forall i \ 1 \le i \le m,\ |S_i|=3$ and has profit $c(S_i)$, find a subset of these sets that maximizes the profit, while each element is covered at most once. Consider a straightforward integer linear programming formulation of the problem:

$$
\max \sum_{i=1}^m c(S_i)x_i
$$
$$
\sum_{i|u\in S_i} x_i \le 1 \quad \forall u \in U
$$
$$
x_i \in \{0, 1\} \quad 1 \le i \le m
$$

Consider the following algorithm that, first computes an optimal solution $x^*$ of the LP obtained by relaxing $x_i \in [0, 1]$, and second performs the following rounding algorithm:
\begin{enumerate}
	\item Chose a set $S_i$ to be in the solution with probability $\frac{x_i^*}{6}$
	\item If an element $u\in U$ is covered by more than one set, remove all the sets in the solution that contain $u$.
\end{enumerate}

Show that the proposed algorithm is a randomized 12-approximation for \texttt{Set packing}

\textbf{Solution:}

Primer de tot, fem les definicions següents:
\begin{gather*}
	C_i := c(S_i) \\
	W := \sum_{i=1}^{m} C_i x_i \\
	opt := \sum_{i=1}^{m} C_i x_{i,opt} \\
	X_i': \text{v.a. indicadora de $x_i$ just després del primer pas de la relaxació} \\
	X_i'': \text{v.a. indicadora de $x_i$ després del segon pas de la relaxació}
\end{gather*}

De la resolució del problema relaxat $(x_i \in [0,1])$, obtenim:
\begin{gather*}
	x_i^* = P[x_{i,opt} = 1] \\
	P[X_i' = 1] = \frac{x_i^*}{6} = \frac{P[x_{i,opt} = 1]}{6} \\
\end{gather*}

Com que el nostre objectiu és demostrar que l'algorisme és una ``12-aproximació aleatoritzada'' del problema, busquem una relació entre l'esperança de $W$ i $opt$. Comencem relacionant l'esperança de $W$ amb la probabilitat de que $X_i'' = 0$.
\begin{gather*}
	E[W] = \sum_{i=1}^{m} C_i E[X_i''] = \sum_{i=1}^{m} C_i P[X_i'' = 1] = \sum_{i=1}^{m} C_i (1 - P[X_i'' = 0])
\end{gather*}

Tot seguit trobem $P[X_i'' = 0]$ en funció de $x_i^*$. La probabilitat que un subconjunt no acabi sent agafat després del segon pas de l'algoritme d'arrodoniment és igual a la probabilitat que no hagi estat agafat en el primer pas més la probabilitat que hagi estat agafat en el primer pas i eliminat en el segon pas (i.e. que existeix un altre subconjunt que també ha estat agafat en el primer pas i comparteix algun element amb el primer subconjunt):
\begin{gather*}
	P[X_i'' = 0] = P[X_i'=0] + P[\exists j : X_i'=1 \land X_j'=1 \land (S_i \cap S_j \neq \emptyset)] \\
	P[X_i'=0] = 1 - P[X_i'=1] = 1 - \frac{x_i^*}{6} \\
	P[\exists j : X_i'=1 \land X_j'=1 \land (S_i \cap S_j \neq \emptyset)] = P\left[
	X_i'=1 \land \bigcup_{S_i \cap S_j \neq \emptyset} X_j'=1
	\right] \\
	P[X_i'=1] \cdot P\left[\bigcup_{S_i \cap S_j \neq \emptyset} X_j'=1
	\right] \le \frac{x_i^*}{6} \cdot \sum_{S_i \cap S_j \neq \emptyset} \frac{x_j^*}{6} \le \frac{x_i^*}{6} \cdot \frac{3}{6} = 
	\frac{1}{12}x_i^*
\end{gather*}

Si ajuntem, trobem que:
\begin{gather*}
	P[X_i'' = 0] \leq 1 - \frac{1}{6}x_i^*  + \frac{1}{12}x_i^* = 1 - \frac{1}{12}x_i^* \\
	P[X_i'' = 1] \geq \frac{1}{12} x_i^* = \frac{1}{12} P[x_{i,opt} = 1]
\end{gather*}

Per tant ara es pot veure que:
\begin{gather*}
	E[W] = \sum_{i=1}^m C_iE[X_i''] = \sum_{i=1}^m C_iP[X_i''=1] \geq \frac{1}{12} \sum_{i=1}^m C_i P[x_{i,opt}]
\end{gather*}

Pel que queda demostrat que aquesta relaxació és una 12-aproximació de \texttt{Set Packing Problem}

\end{document}