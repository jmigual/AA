\documentclass[a4paper]{article}
\usepackage[margin=2cm]{geometry}

\usepackage{fontspec}
\usepackage[catalan]{babel}
\usepackage{algorithm}
\usepackage{algpseudocode}
\usepackage{algorithmicx}
\usepackage{amsmath}
\usepackage{float}

\setlength{\parindent}{0em}
\setlength{\parskip}{1em}

\title{Entrega 1 \\ \textsc{Ampliació d'Algorísmia}}
\author{Marc Asenjo \and Arnau Canyadell \and Joan Marcè}
\date{}

\begin{document}
\maketitle

\section{Problema 14: Geography game}


\emph{One way to pass the time during long journey is to play the word game
of Geography. The first player names a place, say Barcelona. The second player replies
with a place whose name starts with the same letter the previous place ended with, such as
Athens. The first player could then respond with Sicily, and so on. Each place can only be
named once, and the first player who cannot make a move loses.}

\emph{We can make a mathematical version of this game. We have a directed graph $G = (V;E)$ ,
and a designated start node $s \in V$ . The nodes represent the places and the directed edges
represent the legal moves. We start at $s$, and you and I take turns deciding which step to
do next. Whoever ends up in a dead end, where every outgoing edge points to a vertex we
have already visited, loses. A position or node $v$ is a winning position for the first player if
whatever legal move the second player selects, the first player never ends up in a dead end.
geography.}

\begin{itemize}
	\item \emph{Input: An directed graph $G = (V, E)$ and a start node $s$.}
	\item \emph{Question: Is $v$ a winning position for the first player? (The first player can force a win)}
\end{itemize}

\emph{Show that \textsc{geography} is in \texttt{PSPACE}. (In fact \textsc{geography} is a well known \texttt{PSPACE-complete}
problem).}

El següent algoritme resol el problema GEOGRAPHY. Les variables que utilitza són $G$, $v$ i $w$. $G$ té mida $O(|V|+|E|)$ i $v$ i $w$ tenen mida constant. L'alçada de l'arbre de crides recursives generades per l'algoritme està acotada superiorment per $|V|$, i per tant, suposant una implementació (no molt eficient) de l'algoritme en què en cada crida recursiva es genera una còpia del graf que es destrueix a l'acabar la crida, l'espai usat per l'algoritme seria $O(|V|(|V|+|E|))$.

\begin{algorithm}[H]
	\caption{Algoritme per calcular la solució de \textsc{geography}}
	\begin{algorithmic}[1]
		\State // Retorna cert si $v$ és un vèrtex guanyador
		\Function{GEOGRAPHY}{$G = (V, E), v$}
			\State \Return $\forall w | (v,w) \in E : \ \neg$ \Call{GEOGRAPHY}{$G - v$, $w$}
		\EndFunction
	\end{algorithmic}
\end{algorithm}

\pagebreak
\section{Problema 15: Geography game with repeat visits}
\emph{Suppose we play geography without the restriction that we can only visit each vertex once. However, the graph is still directed, and you lose if you find yourself at a vertex with no outgoing edges. Show that the problem of telling whether the first player has a winning strategy in this case is in P. (Notice that if a node is reached after a number of turns greater than $n − 1$, then, for sure there is a draw).}

El següent algorisme és semblant a la del problema 14 però amb algunes diferències. Primer de tot, si es troba un cicle llavors hi ha la possibilitat d'empat. El fet que hi puguin haver empats també canvia la definició recursiva, ja que s'ha d tenir en compte que un jugador intentarà guanyar, i si no pot, intentarà empatar.

Per convenció, decidim que el problema Geography el que vol buscar és opcions guanyadores, i que per tant no vol trobar empats. Per això, quan trobem un cicle farem que el guanyador sigui el jugador que no ha iniciat l'algoritme (i per tant dependrà de si n és parella o senar).

Noti's que la crida inicial de l'algorisme seria amb $d=0$.

\begin{algorithm}[H]
	\caption{Algoritme per calcular la solució de \textsc{geography\_repeat}}
	\begin{algorithmic}[1]
		\State $G = (V, E)$, amb $n = |V|$
		\State $W \in \{0,1,$'?'$\}^{n\times (n+1)}$

		\Function{GEOGRAPHY\_REPEAT}{$v, d$}
			\If{$W(v,d) = $'?'}
				\If{$d > n - 1$}
					\State $W(v,d) := d \mod 2$
				\Else
					\State $W(v,d) := \forall w | (v,w) \in E : \ \neg$ \Call{GEOGRAPHY\_REPEAT}{$w$, $d+1$}
				\EndIf
			\EndIf
			\State \Return $W(v,d)$
		\EndFunction
	\end{algorithmic}
\end{algorithm}

Com que el graf no canvia, l'algorisme es podria implementar utilitzant programació dinàmica, guardant-nos el resultat per a cada node del graf. Per tant com a màxim es faran $|V|$ crides a l'algorisme i en conseqüència el problema pertany a $P$.

\end{document}