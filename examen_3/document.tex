\documentclass[a4paper]{article}

\usepackage[margin=3cm]{geometry}
\usepackage{fontspec}
\usepackage{amsmath, amsfonts}
\usepackage{enumitem}

\setlength{\parindent}{0pt}
\setlength{\parskip}{0.5em}
\def\arraystretch{1.5}

\title{A public-key criptosystem}

\begin{document}
\maketitle
\section*{Statement}
Let us consider the following public-key cryptosystem based on \textsc{Subset Sum}:

\begin{enumerate}[label=\alph*)]
	\item The \emph{private key} consists of a \emph{superincreasing} sequence $E = (e_1, ..., e_n)$ of integers, an integer $m$ greater that the sum of all the elements $e_1, ..., e_n$ and an integer $w$ that is relatively prime to $m$. (A sequence $e_1, ..., e_n$ is called \emph{superincreasing} if each element $e_i$ is strictly greater than the sum of all previous elements).
	\item The \emph{public key} is a sequence $H = (h_1, ..., h_n)$ derived from the private key via $h_i = (we_i) \pmod{m}$.
	\item The encryption of an $n$-bit message $X = (x_1, ..., x_n)$ is the number $c = HX = \sum_{i=1}^n h_ix_i$.
	\item Decrypting the message amounts to solving $c = HX$ for $X \in \{0, 1\}^n$, which is of course equivalent to solving \textsc{Subset Sum} for H with target sum $c$. The owner of the private key, however, can simplify the decryption considerably by calculating $w^{-1}c \equiv_m EX$.
	
	(Now the condition $m > \sum_{i=1}^n e_i$ allows us to replace '$\equiv_m$' by '$=$')
\end{enumerate}

\section*{Solution}
Let us see that the problem of, given an encrypted message $c$ and a private key $\langle E = (e_1, ..., e_n), m, w \rangle$, decrypting $c$, can be solved in polynomial time.
\begin{enumerate}[label=\roman*)]
	\item Show that $w^{-1}c = EX$.
\end{enumerate}

\begin{enumerate}[resume, label=\roman*)]
	\item Prove that the following problem is a polynomial time computable:
	
	Given integers $w$ and $m$ satisfying \emph{a)}, and an integer $c$ satisfying \emph{c)}, compute $(w^{-1}c) \pmod{m}$.
\end{enumerate}
Taking into account that $(w^{-1}c)=_m(w^{-1}\mod{m})·(c \mod{m})$, we know we can compute $w^{-1}\mod{m}$ in $O(n^3)$ using the Extended Euclid algorithm:

 \textbf{EXT-EUCLID}(w,m) returns (d,x,y) where $d=GCD(w,m)=wx + my=1$ because we know $w$ and $m$ are relatively prime. Working in modular arithmetic $my=_m0$, so $wx=_m 1$, and as a result $x=_m w^{-1}$.
 
 Being $c$ a number, computing $c \mod{N}$ is also polynomial, and afterwards the only step left is to compute the product of the two results previously obtained and apply the module one last time in $(w^{-1}c) \pmod{m}$, which are two polynomial operations too. AND WE ARE DONE.

\begin{enumerate}[resume, label=\roman*)]
	\item Show that \textsc{Subset Sum} can be solved in polynomial time if the input sequence of integers is superincreasing. Recall that \textsc{Subset Sum} is defined by:
	
	Given a sequence of integers $e_1, ..., e_n$, and a target integer $W$, compute $X = \{0, 1\}^n$ such that $W = EX$ where $E = (e_1, ..., e_n)$. (\emph{Hint}: be greedy).
	
	Hence, given the superincreasing sequence $E$ and $W = (w^{-1}c) \pmod{m}$, we can compute the decrypted original message $X$ in polynomial time.
\end{enumerate}



\end{document}