\documentclass[a4paper]{article}

\usepackage[margin=3cm]{geometry}
\usepackage{fontspec}
\usepackage{amsmath, amsfonts}
\usepackage{enumitem}
\usepackage{algorithm}
\usepackage{algorithmicx}
\usepackage{algpseudocode}
\PassOptionsToPackage{hyphens}{url}\usepackage{hyperref}

\setlength{\parindent}{0pt}
\setlength{\parskip}{0.5em}
\def\arraystretch{1.5}

\algnewcommand\algorithmicinput{\textbf{Input:}}
\algnewcommand\INPUT{\item[\algorithmicinput]}
\algnewcommand\algorithmicoutput{\textbf{Output:}}
\algnewcommand\OUTPUT{\item[\algorithmicoutput]}	

\title{A public-key criptosystem}
\author{Arnau, Joan \& Marc  <3}

\begin{document}
\maketitle
\section*{Statement}
Let us consider the following public-key cryptosystem based on \textsc{Subset Sum}:

\begin{enumerate}[label=\alph*)]
	\item The \emph{private key} consists of a \emph{superincreasing} sequence $E = (e_1, ..., e_n)$ of integers, an integer $m$ greater that the sum of all the elements $e_1, ..., e_n$ and an integer $w$ that is relatively prime to $m$. (A sequence $e_1, ..., e_n$ is called \emph{superincreasing} if each element $e_i$ is strictly greater than the sum of all previous elements).
	\item The \emph{public key} is a sequence $H = (h_1, ..., h_n)$ derived from the private key via $h_i = (we_i) \pmod{m}$.
	\item The encryption of an $n$-bit message $X = (x_1, ..., x_n)$ is the number $c = HX = \sum_{i=1}^n h_ix_i$.
	\item Decrypting the message amounts to solving $c = HX$ for $X \in \{0, 1\}^n$, which is of course equivalent to solving \textsc{Subset Sum} for H with target sum $c$. The owner of the private key, however, can simplify the decryption considerably by calculating $w^{-1}c \equiv_m EX$.
	
	(Now the condition $m > \sum_{i=1}^n e_i$ allows us to replace '$\equiv_m$' by '$=$')
\end{enumerate}

\section*{Solution}
Let us see that the problem of, given an encrypted message $c$ and a private key $\langle E = (e_1, ..., e_n), m, w \rangle$, decrypting $c$, can be solved in polynomial time.
\begin{enumerate}[label=\roman*)]
	\item Show that $w^{-1}c = EX$.
\end{enumerate}

\begin{enumerate}[resume, label=\roman*)]
	\item Prove that the following problem is a polynomial time computable:
	
	Given integers $w$ and $m$ satisfying \emph{a)}, and an integer $c$ satisfying \emph{c)}, compute $(w^{-1}c) \pmod{m}$.
\end{enumerate}

\begin{enumerate}[resume, label=\roman*)]
	\item Show that \textsc{Subset Sum} can be solved in polynomial time if the input sequence of integers is superincreasing. Recall that \textsc{Subset Sum} is defined by:
	
	Given a sequence of integers $e_1, ..., e_n$, and a target integer $W$, compute $X = \{0, 1\}^n$ such that $W = EX$ where $E = (e_1, ..., e_n)$. (\emph{Hint}: be greedy).
	
	Hence, given the superincreasing sequence $E$ and $W = (w^{-1}c) \pmod{m}$, we can compute the decrypted original message $X$ in polynomial time.
\end{enumerate}

The following greedy algorithm solves the problem in linear time.

\begin{algorithm}[H]
	\caption{\textsc{Subset Sum} with superincreasing list}
%	\KwData{}
%	\KwResult{  }
\begin{algorithmic}
	\INPUT $E= (e_1, ... , e_n), W $
	\OUTPUT $ X = \{0,1\}^n $ such that $W=EX$
	\State \Comment Each variable $x_j$ is boolean in spirit: it indicates if you include $e_j$ in the sum or not.
	\State $s := 0$ \Comment Sum variable
	\State $j := n$ \Comment Index variable
	\While{$ j>0 $ \and $ s<W $}
		\If{$ s + e_j \leq W $} \Comment Ifincluding  $e_j$ will not create overflow, we include it.
			\State $s := s + e_j$
			\State $x_j := 1$
		\Else
			\State $x_j := 0$
		\EndIf
		$ j := j-1 $
	\EndWhile
\end{algorithmic}
\end{algorithm}

Reference: algorithm explained here \rightarrow \url{https://math.stackexchange.com/questions/378148/greedy-optimized-subset-sum-problem}.

This is a greedy algorithm that solves the \textsc{Subset Sum} problem without guaranteeing an optimal solution (assuming that $E$ is still an ordered sequence). However, if the input list is superincreasing, the optimal is guaranteed.

\end{document}